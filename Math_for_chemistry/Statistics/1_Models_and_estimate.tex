%usual heads
\documentclass[12pt]{ctexart}%中文包
\usepackage{graphicx}%图片引用包
\usepackage{amssymb}%特殊符号
\usepackage{amsmath,amsfonts,bm}%矩阵
\usepackage{amsthm}%定理
\usepackage{geometry}%缩放
\usepackage{hyperref}%超链接
\usepackage{framed}%框框
\usepackage{color}%颜色
\usepackage{mathrsfs}%花体
\usepackage{anyfontsize}
\usepackage{indentfirst}%缩进
\usepackage{extsizes}%size
\usepackage{newtxtext,newtxmath}%times风格字体
\usepackage{mdframed}%边栏


\surroundwithmdframed[
  linecolor=black,
  linewidth=1pt,
  innerleftmargin=10pt,
  innerrightmargin=10pt,
  innertopmargin=0pt,
  innerbottommargin=10pt
]{thm}
\surroundwithmdframed[
  linecolor=gray,
  topline=false,
  bottomline=false,
  rightline=false,
  linewidth=4pt,
  innerleftmargin=10pt,
  innerrightmargin=10pt,
  innertopmargin=0pt,
  innerbottommargin=5pt
]{df}
\surroundwithmdframed[
  linecolor=black,
  linewidth=1pt,
  innerleftmargin=10pt,
  innerrightmargin=10pt,
  innertopmargin=0pt,
  innerbottommargin=10pt
]{psn}

\newcommand\R {\mathbb{R}}%简写
\newcommand\E {\mathbb{E}}
\newcommand\Pd {\mathbb{P}}
\newcommand\tn {\hat{\theta}_n}

\setlength{\parindent}{2em}%缩进参数
\linespread{1.5}%行间距
\geometry{a4paper,scale=0.75}%大小
\newtheorem{thm}{Theorem}%定理环境
\newtheorem{df}{Definition}
\newtheorem{psn}[thm]{Proposition}

\hypersetup
    {
        hypertex=true,
        colorlinks=true,
        linkcolor=blue,
        anchorcolor=blue,
        citecolor=blue
    }%超链接

\title{Statistic Models and Estimation\\统计模型和估计}
\author{Jerry Ling} 
\date{\today}
 


\begin{document}  %begin后括号内是文件类型
\maketitle  %生成标题
概率论由随机变量性质研究其结果,而数理统计由有限观测结果研究随机变量本身性质。为了建立其框架,首先引入统计模型的概念:
\begin{df}[Statistic Model]
    Suppose our observed outcomes $\{X_i\}$s is generated by r.v. on probability space (E,$\mathscr{B}$,\rm{P}) described by distribution $\mathbb{P}$, a statistic model of this experiment is a pair 
    \[(E,(\mathbb{P}_{\theta})_{\theta \in \Theta })\]
    i.e. a collection of distributions on space E, each described by parameter $\theta$.
\end{df}
注意我们选取的模型未必能代表随机变量。称能代表之的模型(存在$\theta$使得$\mathbb{P}_{\theta}=\mathbb{P}$)为\textbf{well specified},正确的参数为\textbf{真参数},此后讨论\underbar{默认其存在}。当参数空间$\Theta$有限维($\in \mathbb{R}^d$)称为\textbf{参数化(parametric)模型}。在某些时候我们会直接估计分布函数(及其泛函),因为其属于函数空间,故是非参数估计。线性回归方法则使用与n个参数来建立(n+1)维数据的关系,因而是参数模型。
\par 另一方面,为了清晰性,我们还要求参数到分布的映射是单射的,即模型的参数是\textbf{可分辨的(identifible)}。这是为了避免多个真参数的情况。指认了模型后,为了找到这个真参数,我们需要利用观测得到的n个数据,即n个独立等同分布的随机变量{$X_i$}:
\begin{df}[Statictics and Estimator]
    A statistics(统计量) is a function $g$ of n observed r.v.
    $\{X_i\}_{i=1,2,...,n}$.
    An estimator $\hat{\theta}_n$ is a statistics that \textbf{does not} contain $\theta$.
    \[ \tn \triangleq g(\{X_i\}_{i=1,2,...,n})\]
\end{df}
\par 注意,估计器本身是一个随机变量(序列),且必然由真分布生成。因而我们可以计算其期望、方差(序列)并定义其偏差和标准差:
\begin{equation}
    bias\triangleq\mathbb{E}\hat{\theta}_n-\theta
\end{equation}
\begin{equation}
    se(\hat{\theta}_n)\triangleq\sqrt{\mathbb{V}ar \hat{\theta}_n}
\end{equation}
此后标准差\underbar{简写为se}。若估计器序列收敛到我们需要的参数,则称之为\textbf{一致的(consistent)}:
\begin{thm}[Consistency Condition]$\\$
    If \[bias \to 0,se\to 0 \]
    then \[MSE=\E(\hat{\theta}-\theta)^2 \to 0 \]
    then \[\hat{\theta}\xrightarrow{\Pd}\theta\]
\end{thm}
其中MSE是\textbf{均方差(mean squared error)},等于偏差和标准差的平方和。
\par 考虑到每个观测量都是独立等同分布,我们容易想到使用中心极限定理来刻画较大观测数量(30+)的估计器的分布。实际上,如果能说明估计器序列本身在真值附近是\textbf{渐近正态分布(asymptotically normal)},就能直接导出该估计的置信区间:
\section*{Confidence Interval}
我们的问题是,当模型和估计器已知时,如何(近似)导出置信区间。首先我们考虑0-1上的Ber(p)变量,p的估计器$\hat{p}_n$设定为其均值。根据大数定律其必然收敛于其期望,即偏差趋于0。记Ber(p)的方差为$\sigma^2=p(1-p)$,则:
\begin{equation}
    se=\frac{\sigma}{\sqrt{n}}
\end{equation}
显然se也趋于0,则该估计器是一致的。根据中心极限定理$\frac{\hat{p}_n-p}{se}$收敛到正态分布,也就是说此处估计器本身就是渐进正态分布:
\begin{equation}
    \sqrt{n}\frac{\hat{p}_n-p}{\sqrt{p(1-p)}}\xrightarrow{d.}N(0,1)
    \label{converge_1}
\end{equation}
\par 此处估计器$\hat{p}_n$是随机的,而真参数p是确定值。为估计之,考虑在$\hat{p}_n$周围构造一个区间——其位置参数$p_1$,$p_2$也是随机变量,构造条件为该区间囊括真参数p的概率$1-\alpha$,也就是\textbf{置信度(level of confidence)}:
\begin{df}[Confidence Interval]
    At level of confidence 1-$\alpha$, $CI\triangleq(\hat{p}_n-p_1,\hat{p}_n+p_2)$, such that:
    \[\rm{P}_p(\hat{p}_n-p_1<p<\hat{p}_n+p_2)\geq 1-\alpha\]
\end{df}
其中P的下标表示其中的随机变量均由真参数分布$\Pd_p$生成。将其同(\ref{converge_1})比较发现只要分母上的p可以去除掉,那么就可以根据高斯分布给出置信区间。其中又可以使用\textbf{保守上界(conservative bound)}、\textbf{插入(plug-in)}、\textbf{解方程(solving equations)}三种方法得到。此例子中,使用上界是较好的方法。插入法即将方差p(1-p)的估计器$\hat{\sigma}_n^2$直接替换之,在n较大时有效。
\section*{Delta Method for CI Determination}
再考虑一个常见的分布Exp($\lambda$),因为期望是参数$\lambda$的倒数,估计器可如下设计:
\begin{equation}
    \hat{\lambda}_n=\frac{1}{\frac{1}{n}\sum X_i}
\end{equation}
由于大数定律,分母依概率收敛于期望,因而这也是一个\textbf{无偏的(unbiased)}估计器。但中心极限定理却不能直接运用到估计器上:
\begin{equation}
    \sqrt{n}\frac{1/\hat{\lambda}_n-1/\lambda}{\sigma}=\sqrt{n}\frac{1/\hat{\lambda}_n-1/\lambda}{1/\lambda}\xrightarrow{d.}N(0,1)
    \label{converge_2}
\end{equation}
因而引入一个实用的定理:
\begin{thm}[Delta Method]For a function g: $\\$
    If \[ Y_n \xrightarrow{\Pd} N(\mu,\frac{\sigma^2}{n}),g'(\mu)\neq 0\]
    Then \[g(Y_n)\xrightarrow{\Pd} N(g(\mu),g'(\mu)^2\frac{\sigma^2}{n})\]
\end{thm}
则代入反比例函数以导出式(\ref{converge_2})中估计器$\hat{\lambda}_n$的分布:
\begin{equation}
    \hat{\lambda}_n \xrightarrow{d.} N(\lambda,\lambda^4\frac{\sigma^2}{n})=N(\lambda,\frac{\lambda^2}{n})
\end{equation}
其后置信区间的计算与上一节介绍的流程基本一致,但方程法从二元变为线性,而上界法失效。
\section*{Method of Moment}

\end{document}