\documentclass[12pt]{ctexart}%中文包
\usepackage{graphicx}%图片引用包
\usepackage{amssymb}%特殊符号
\usepackage{amsmath,amsfonts,bm}%矩阵
\usepackage{amsthm}%定理
\usepackage{geometry}%缩放
\usepackage{hyperref}%超链接
\usepackage{framed}%框框
\usepackage{color}%颜色
\usepackage{mathrsfs}%花体
\usepackage{anyfontsize}
\usepackage{indentfirst}%缩进
\usepackage{extsizes}%size
\usepackage{newtxtext,newtxmath}%times风格字体
\usepackage{mdframed}%边栏

\surroundwithmdframed[
  linecolor=gray,
  topline=false,
  bottomline=false,
  rightline=false,
  linewidth=4pt,
  innerleftmargin=10pt,
  innerrightmargin=10pt,
  innertopmargin=0pt,
  innerbottommargin=10pt
]{theorem}
\surroundwithmdframed[
    linecolor=black,
    leftline=false,
    rightline=false,
    linewidth=0.5pt,
    innerleftmargin=10pt,
    innerrightmargin=10pt,
    innertopmargin=0pt,
    innerbottommargin=10pt
]{note}

\newtheorem{theorem}{Law}
\newtheorem{note}{Note}



\geometry{a4paper,scale=0.85}
\hypersetup
    {
        hypertex=true,
        colorlinks=true,
        linkcolor=blue,
        anchorcolor=blue,
        citecolor=blue
    }


\title{Introductioin Single Molecule Science\\单分子科学绪论}

\author{Jerry Ling} 

\date{\today}
 
\begin{document}  %begin后括号内是文件类型
\maketitle  %生成标题
本笔记系张云翔老师开设的研究生课程《单分子科学:原理与模型》的课程笔记。主要参考书 \textit{Single Molecule Science},可参考 \textit{An Introduction to Single Molecule Biophysics} by Yuri L. Lyubchenko (偏向生物物理),\textit{Single Molecule Techniques} by Paul R. Selvin and Taekjip Ha (实验细节,fret), \textit{Single Molecule Spectroscopy in Chemistry, Physics and Biology} (较经典的论文集)。

\section*{单分子科学前传}
\subsection*{从朴素原子论到近代原子论}
古希腊人提出原子和真空的最初的概念。思辨上,他们认为物体要运动,中间就需要是真空(that do not exist),否则就会被阻挡。17世纪,波义耳在他的文章中提出corpuscle(unbroken particles)的概念,着重强调不同原子的\textbf{排列组合}能够产生新物质。某种意义上,corpuscle就是particle的原型。
\par 原子论更进一步的发展依赖于对气体的研究和\textbf{实验}(Gay-Lussac)。阿伏伽德罗则作为理论家统合了道尔顿的原子论和前述的实验,提出了分子理论(molecule theory, 1811)。丹尼尔·伯努利则根据动量守恒导出了分子的平均速率,首次从bulk水平上理解分子的运动情况。在微观尺度上,Loschmidt是第一个对分子大小进行估计的科学家(1865)。他通过凝聚系数(condensation coefficient)和气体自由程相当准确地估计出气体分子的大小。
\begin{equation}
    \bar{v}=\sqrt{\frac{8kT}{\pi m}} \ \ \ \ v_p=\sqrt{\frac{2kT}{m}}
\end{equation}
\begin{framed}
    Exercise: Calculate the speed of Ammonia and compare it with the speed of a shot bullet.
\end{framed}
\begin{note}[the Bernoulli Family]
    John Bernoulli, known for Catenary Curve, Brachistochrone Cruve and mentor of Euler. Jakob Bernoulli, who is John's brother, is known for his achievement in probability theory. Daniel Bernoulli, Old John's son, is known for his research in hydrodynamics. He is the one we've mentioned above.
\end{note}
\subsection*{向单分子迈进}
尽管19世纪的科学家完成了很多出色的工作,却仍未有实验\textbf{直接证明}分子的实体存在,直到爱因斯坦对布朗运动进行了系统的研究。1827年布朗观察到花粉在水面上的随机运动,而1905年爱因斯坦导出了扩散速率和热力参数的关系:
\begin{equation}
    mse = \frac{RT}{3\pi\eta N_{\alpha} r}\tau = 2 D \tau
\end{equation}
\begin{framed}
    Extra Reading: the BM theory is confirmed in 1908, revisited in 2008. 
\end{framed}

\end{document}