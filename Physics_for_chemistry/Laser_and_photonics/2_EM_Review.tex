\documentclass[12pt]{ctexart}%中文包
\usepackage{graphicx}%图片引用包
\usepackage{amssymb}%特殊符号
\usepackage{amsmath,amsfonts,bm}%矩阵
\usepackage{amsthm}%定理
\usepackage{geometry}%缩放
\usepackage{hyperref}%超链接
\usepackage{framed}%框框
\usepackage{color}%颜色
\usepackage{mathrsfs}%花体
\usepackage{anyfontsize}
\usepackage{indentfirst}%缩进
\usepackage{extsizes}%size
\usepackage{newtxtext,newtxmath}%times风格字体
\usepackage{mdframed}%边栏

\surroundwithmdframed[
  linecolor=gray,
  topline=false,
  bottomline=false,
  rightline=false,
  linewidth=4pt,
  innerleftmargin=10pt,
  innerrightmargin=10pt,
  innertopmargin=0pt,
  innerbottommargin=10pt
]{theorem}
\surroundwithmdframed[
    linecolor=black,
    leftline=false,
    rightline=false,
    linewidth=0.5pt,
    innerleftmargin=10pt,
    innerrightmargin=10pt,
    innertopmargin=0pt,
    innerbottommargin=10pt
]{note}

\newtheorem{theorem}{Law}
\newtheorem{note}{Note}



\geometry{a4paper,scale=0.85}
\hypersetup
    {
        hypertex=true,
        colorlinks=true,
        linkcolor=blue,
        anchorcolor=blue,
        citecolor=blue
    }


\title{Electromagnetic Theory\\激光电磁理论}

\author{Jerry Ling} 

\date{\today}
 
\begin{document}
\maketitle
大部分电磁基础理论的处理参见Optics的第三篇笔记。下面讨论一些不一样的approach。
\subsection*{傅立叶变换后的Maxwell方程}
对其作傅立叶变换,假定线性介质,那么取一个频率得到仅包含空间关系的方程:
\begin{equation}
    \nabla \times \textbf{H}(\textbf{r}) = \textbf{J}(\textbf{r})+j\omega\textbf{D}(\textbf{r})
\end{equation}
\begin{equation}
    \nabla \times \textbf{E}(\textbf{r}) = -j\omega\mu_0\textbf{H}(\textbf{r})
\end{equation}
其中极化矢量\textbf{P}本质上源于电场作用下介质的响应,与电位移的关系定义为:
\begin{equation}
    \textbf{D}(\textbf{r}) = \varepsilon_0\textbf{E}(\textbf{r})+\textbf{P}(\textbf{r})
\end{equation}
可用物质方程和电场联系:
\begin{equation}
    \textbf{P} = \epsilon_0\chi_e\textbf{E}
\end{equation}
\par 当我们使用真空平面波条件,不难简化得到波矢、电场和磁场的关系:
\begin{equation}
    \textbf{k}\times\textbf{H}=-\omega\epsilon_0\textbf{E}
\end{equation}
\begin{equation}
    \textbf{k}\times\textbf{E}=\omega\mu_0\textbf{H}
\end{equation}
能得到一个有趣的关系,\textbf{真空阻抗}(377 ohm):
\begin{equation}
    \frac{|\textbf{E}|}{|\textbf{H}|} = \sqrt{\frac{\mu_0}{\epsilon_0}}
\end{equation}
\subsection*{波动方程延伸}
\par 激光器中的波动方程可写作\textbf{含源的非齐次方程}:
\begin{equation}
    \nabla^2\textbf{e}-\frac{n^2}{c^2}\partial_t^2\textbf{e}=\mu_0\partial_t^2\textbf{p}_{\epsilon}
\end{equation}
注意此时介质的贡献整合进n中,而激活原子(active atom)对场的贡献充当\textbf{场源(source)}。
\subsection*{不确定性和光束发散}
下面讨论一下光斑的发散,这也就涉及到不确定性原理。简而言之,由于波矢和位置的傅立叶对易关系,位置的分布越窄,波矢(即传播方向)的分布就越宽。
\begin{equation}
    \Delta k_x\Delta x \geq \frac{1}{2}
\end{equation}
换句话说,越大的出射口径往往带来更小的发散角。这个关系允许我们对光束进行傅立叶分析,甚至看作一系列平面波的传播(角谱)。这在光学中极为重要。
\subsection*{界面光学}
\textbf{布鲁斯特角(Brewster's Angle)}(p-波,即面内波在折射垂直反射时无法反射的情况):
\begin{equation}
    \tan \theta_1 = \frac{n_2}{n_1}
\end{equation}
在光学中,介质常常被切割成布鲁斯特角以减少反射(比镀膜要耐久),尤其是在高能激光器中。
\end{document}