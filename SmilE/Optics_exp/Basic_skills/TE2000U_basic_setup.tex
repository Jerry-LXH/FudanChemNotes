\documentclass[12pt]{ctexart}%中文包
\usepackage{graphicx}%图片引用包
\usepackage{amssymb}%特殊符号
\usepackage{amsmath,amsfonts,bm}%矩阵
\usepackage{amsthm}%定理
\usepackage{geometry}%缩放
\usepackage{hyperref}%超链接
\usepackage{framed}%框框
\usepackage{color}%颜色
\usepackage{mathrsfs}%花体
\usepackage{anyfontsize}
\usepackage{indentfirst}%缩进
\usepackage{extsizes}%size
\usepackage{newtxtext,newtxmath}%times风格字体
\usepackage{mdframed}%边栏

\surroundwithmdframed[
  linecolor=gray,
  topline=false,
  bottomline=false,
  rightline=false,
  linewidth=4pt,
  innerleftmargin=10pt,
  innerrightmargin=10pt,
  innertopmargin=0pt,
  innerbottommargin=10pt
]{theorem}
\surroundwithmdframed[
    linecolor=black,
    leftline=false,
    rightline=false,
    linewidth=0.5pt,
    innerleftmargin=10pt,
    innerrightmargin=10pt,
    innertopmargin=0pt,
    innerbottommargin=10pt
]{note}

\newtheorem{theorem}{Law}
\newtheorem{note}{Note}



\geometry{a4paper,scale=0.85}
\hypersetup
    {
        hypertex=true,
        colorlinks=true,
        linkcolor=blue,
        anchorcolor=blue,
        citecolor=blue
    }


\title{TE2000U 基本使用笔记}

\author{Jerry Ling} 

\date{\today}
 
\begin{document}
\maketitle
A1043有两台科研级的尼康倒置显微镜,进门右手边(TE2000U)是组里最常用的荧光TIRF(全内反射)配置,示意图如下。本笔记针对A1043的\textbf{一般使用者},包括基本成像光路、如何正确地上样测试和日常维护,以避免初用者踩坑,并让仪器维持一个可持续的状态。目前更换光纤、调整光路、更换(不是切换)二相色镜/滤镜、更换双通道都需要授权进行,如有需要可联系张老师学习操作,有关文档另作整理。关于光学间本身和进入前后的注意事项,参考\textit{Clean Room Basics \& Rules}和\textit{Clean Room Checklist}。



\section*{光路描述和原理}
\subsection*{激发光路}
\begin{itemize}
    \item 耦合进光纤的激光(B1)经过\textbf{光纤准直器(Fiber Collimator)} (图\ref{collimator})变成一束较小的平行光(其调节方法见另一篇笔记)。
    \item 而后经过一组由两个透镜构成的\textbf{telescope型扩束器},扩束为更大直径的准直光。第一面凹透镜负责扩大光斑,第二面凸透镜负责将其转化为平行光,因而两片透镜间距须恰好为焦距之差(即凸透镜焦点恰好落在凹透镜焦点上)。两个透镜可通过其笼架上的旋钮微调x-y位置,使得光线垂直地通过其轴心,否则将影响后续光路(请勿随意调整)。
    \item 经过telescope扩束器的光束即随距离尺寸几乎不变的平行光。这样再经过一面较长焦距的透镜(400mm,即图\ref{optics}中的49639 INK)、被\textbf{二相色镜(dichroic mirror)}反射后即可将激光较好地汇聚在物镜的\textbf{后焦面(Back Focal Plane)}上(图\ref{optics}中B2)。
\end{itemize}

\begin{figure}[h] 
    \centering
    \includegraphics[width=0.6\textwidth]{fiber-collimators.png}
    \caption{光纤准直器}
    \label{collimator}
\end{figure}

\subsection*{物镜后焦面事件}
\begin{itemize}
    \item 后焦面上聚焦的激光经过物镜变成平行光对样本进行照明。当聚焦点在后焦面正中心时,出射光亦在正中心垂直出射,称为\textbf{落射照明(epi-illuminance)}。此时在物镜口可看到对称分布的一圈圈的圆光斑。
    \item 若后焦面上汇聚的点不在中心,那么出射光将按一个角度照亮样品;当角度大于介质-样品界面的临界角时,发生\textbf{全内反射(TIRF)},反射光(即图\ref{optics}中返回的红线)重新聚焦在后焦面对称的位置。
    \item 可以通过调整telescope中的凸透镜(见图中所示)的旋钮来控制入射点的位置(本质上是控制光线的角度)。由于左右方向比较稳定,建议只调整上下方向来控制TIRF。可以观察物镜出口光斑位置来判断是否发生TIRF,当光斑大部分“潜入”物镜即可认为样品由倏逝波照明,即照明强度随z轴指数衰减。请勿随意调整其他的旋钮。
\end{itemize}

\begin{figure}[h] 
    \centering
    \includegraphics[width=0.9\textwidth]{Microscope_fig.jpeg}
    \caption{TE2000U光路图}
    \label{optics}
\end{figure}

\subsection*{出射光路}
\begin{itemize}
    \item 出射光中包含样品发出的荧光、入射光被反射后形成的反射光、入射光与样品作用后发出的散射光。后两类一般会被二相色镜和\textbf{出射滤镜(emission filter)}去除,只透过前者以观察荧光。(二相色镜对不同波长的光有选择性地反射/透射)
    \item 出射光重新经过物镜,则物镜前焦面上点源发出的光变成平行光;而在物镜后焦面(B2)处恰成一个\textbf{fourier变换像},因而这个面也被称为\textbf{fourier面}(换言之,这个面上的光场分布恰是焦面上光场分布的fourier变换)。为观察之,可在像面后一个焦距的位置用一个伯特兰透镜(Bertrand Lens)对像再进行一次fourier变换(图\ref{optics}中左下角,BFP Imaging Setup)。
    \item 平行光穿过二相色镜后再经过一个\textbf{镜筒透镜(Tube Lens, 200nm)}被重新汇聚在\textbf{像面(Imaging Plane, A2)}上。这个像面一般在显微镜side port出口的几厘米左右。在此面上放置相机即可对焦面成像。
    \item 出射光经过Tube lens后再经过反射镜转到侧口。显微镜右侧旋钮可以切换出射光光路到左侧或右侧port,注意须\textbf{拧到听到click声音}以切换。
\end{itemize}

\section*{搭建正确的激发光路(optional/了解即可)}
\begin{enumerate}
    \item 检查光纤耦合器出来的光斑是否对称,如果不对称则说明激光不准直,可按另一篇笔记内容小心调整xy旋钮。
    \item telescope的两个镜子首先要准直地放大光斑。应看到对称的光斑、且光斑大小应随距离变化不大。这样说明两面镜子与激光同轴且间距正确。
    \item focusing lens首先需要与前面的部件共轴。然后可通过\textbf{笼式系统(cage system)}调整其位置以将激发光汇聚到物镜后焦面。
    \item 当4面透镜都同轴且放置在正确的距离上时,即达成落射照明后,方可按照前述方法调整到TIRF照明。
    \item 由于物镜后焦面基本上与Tube lens前焦面重合,后焦面发出的光像面(A2)处可看作平行光,因而再放一个凸透镜即可对后焦面成像。当发生TIRF时候,反射光应在圆形后焦面的边缘处。
\end{enumerate}

\par 整体的调整思路是:先保证落射照明(epi,平行光激发)、再达成TIRF。改变角度更好的方法是整体移动excitation部分,而非调整一面镜子的位置(这可能会让光线倾斜过多而损失一些光)。一旦调整到较好状态,尽量不要去调整telescope的位置以避免偏轴等复杂问题。

\section*{正确的上样和日常使用规范}
日常测试过程中,需要注意显微镜使用规范和日常维护,以确保测到有意义的数据以及仪器的可持续使用:

\subsection*{光路}
\begin{itemize}
    \item 如需使用显微镜,在确保相机安装正确、未有暴露机械快门的情况下,打开相机电源。打开计算机上的控制软件(Solis, software for Andor EMCCD)。EMCCD需要冷却至-70度来降低噪声,请务必等候约5分钟至左下角温度指示色块\textbf{由红变蓝},再开始采集数据。关于相机的使用和控制软件细节使用请参见另一篇笔记。
    \item 光纤需要拧好,不要过松或过紧(finger tight)。更换光纤需要通知老师学习,详请参考另一篇笔记。
    \item 检查是否使用了1.5x增距镜。
    \item 进光前\textbf{检查二相色镜}是否正确(须在出射光路中滤掉激发光)。切换时候需注意听到click声音代表切换成功。否则,使用错误的二向色镜采集图像时极有可能会过曝严重,此时须\textbf{立刻关闭采集}以保护相机。更换二相色镜,需要通知老师学习,详请参考另一篇笔记和视频。
    \item 调整光斑达成TIRF的过程中避免不必要的调整(见上文)。此外,请注意避免对激发光路施加机械冲击以防止光路歪轴,比如身体不要依靠或者撞击光学平台;不要在黑暗中快速移动肢体。
    \item 眼睛\textbf{不可直视激光}。眼睛与光路不可处于同一水平高度。可代用档光片或纸片观察激光光斑是否正常。尤其注意近红外波段的激光不可见但仍会造成眼部永久损伤,应格外当心。调整光路时务必使用合适的\textbf{防护眼镜}。
    \item 如需更换物镜,请务必确保其下降到安全高度,否则将刮伤物镜。
\end{itemize}

\subsection*{上样}
\begin{itemize}
    \item 上样前请务必确保物镜下降到安全的高度,以避免碰伤物镜。下降过程请避免使用粗调旋钮,使用微调knob的时候请务必耐心慢慢旋转,避免内部齿轮磨损。上油要\textbf{避免气泡},更不要摔落镜油瓶。如油瓶中气泡很多,需要degas。气泡将带来明显的衍射条纹、并随着样品台移动变化。可尝试移动样品台来赶走气泡。
    \item 上样的板子不能过长或过短,以免不对称(翘起)或掉落样品架。coverglass 务必使用24*30或24*32 \#1.5(H),并且组装在25*75 slide正中央。24*40一定无法使用,24*50或者24*60在对称组装下可以使用。如果严重偏离对称,或者coverglass刚好长40mm与样品架开口(40mm)一样大小将导致样品无法正确安装。
    \item 尽量避免于样品安装在样品架上时进行injection(向channel内注射以孵育或清洗),以防溶液与镜油混合而导致无法正常成像。
    \item 确保双面胶(3M)\textbf{压平密封}。若其明显发白,说明溶液有渗漏,最好重新制样以避免无效数据或溶液漏进镜油。
    \item 找焦面不可大幅调整高度,否则有可能撞伤物镜。若发现高度过头可重新调低找焦。
    \item 水平调整样品台时切不可幅度过大,否则其边沿可能撞到物镜。找焦或者换channel时要注意出射激光走向,建议挡住激光再靠近观察。
\end{itemize}

\subsection*{维护}
\begin{itemize}
    \item 实验结束记得关闭软件、相机和激光器。先关闭solis软件,再关闭相机电源。软件非正常关闭,或者顺序颠倒将导致相机再次启动时setup参数丢失,或进入multitrack光谱采集模式。激光器关闭一般遵循:先关电流,然后等若干分钟再关电源的流程。做细胞实验的同学记得关闭明场照明的卤素灯电源。
    \item 光纤脆弱,不可用力拉拽。注意光纤收纳的最小半径(一般>=10cm),过小半径将损坏光纤。
    \item 物镜上的油须擦净。外圈的油可用纸巾(Kimwipe)\textbf{吸走},而不宜大范围地涂抹。玻璃的部份则必须使用\textbf{擦镜纸}以避免纤维划伤表面,可用止血钳夹住一截对折数次的擦镜纸转圈擦拭,注意擦过一次后要换到清洁的部份再擦,以及止血钳金属的部份不要刮到物镜。确保擦完后玻璃\textbf{表面光亮无油渍},若有油渍残留会造成折射率不均而劣化成像,照明光斑也会形态异常。
    \item 避免过量加油,多次加油后必须吸掉重加。吸油,擦油时务必小心,动作幅度不要过大过快,以免将油挤出物镜储油槽,或者溅射到物镜以外,如位移台,转盘等关键部位。绝不可让油满溢到物镜壁上,因为油会损害下游部件。若已流出,立刻擦掉;若已流到螺口则需要通知老师处理。
    \item 结束后记得在记录本上登记本次实验。
\end{itemize}

\section*{待办}
\begin{itemize}
    \item 相机使用规范
    \item 更换滤镜/二相色镜方法规范
    \item 光纤更换和调整方法规范
    \item 双通道测试
    \item 增加一些图片:光路、Filter Cube、软件界面、调tirf图片、光斑图片
\end{itemize}

\end{document}